%!TEX TS-program = xelatex 
%!TEX TS-options = -synctex=1 -output-driver="xdvipdfmx -q -E"
%!TEX encoding = UTF-8 Unicode
%
%  transparency
%
%  Created by Mark Eli Kalderon on 2014-07-08.
%  Copyright (c) 2014. All rights reserved.
%

\documentclass[12pt]{article} 

% Definitions
\newcommand\mykeywords{Aristotle, transparency}
\newcommand\myauthor{Mark Eli Kalderon}

% Packages
\usepackage{geometry} \geometry{a4paper} 
\usepackage{url}
\usepackage{txfonts}
\usepackage{color}
\usepackage{enumerate}
\definecolor{gray}{rgb}{0.459,0.438,0.471}
\usepackage{setspace}
% \doublespace % Uncomment for doublespacing if necessary
% \usepackage{epigraph} % optional

% XeTeX
\usepackage[cm-default]{fontspec}
\usepackage{xltxtra,xunicode}
\defaultfontfeatures{Scale=MatchLowercase,Mapping=tex-text}
\setmainfont{Hoefler Text}

% Bibliography
\usepackage[round]{natbib}

% Title Information
\title{Aristotle on Transparency}
\author{\myauthor} 
\date{} % Leave blank for no date, comment out for most recent date

% PDF Stuff
\usepackage[plainpages=false, pdfpagelabels, bookmarksnumbered, backref, pdftitle={Form Without Matter}, pagebackref, pdfauthor={\myauthor}, pdfkeywords={\mykeywords}, xetex, colorlinks=true, citecolor=gray, linkcolor=gray, urlcolor=gray]{hyperref} 

%%% BEGIN DOCUMENT
\begin{document}

% Title Page
\maketitle
% \begin{abstract} % optional
% \noindent
% \end{abstract} 
% \vskip 2em \hrule height 0.4pt \vskip 2em
% \epigraph{text of epigraph}{\textsc{author of epigraph}} % optional; make sure to uncomment \usepackage{epigraph}

% Layout Settings
\setlength{\parindent}{1em}

% Main Content

In \emph{La Dioptrique}, Descartes makes the striking and paradoxical comparison between vision and a blind man's use of sticks in navigation, a kind of haptic touch. The analogy is, in fact, an ancient one. Alexander of Aphrodisias attributes it to the Stoics (\emph{De Anima} 130 14). The Stoic analogy was criticized by Galen in \emph{De Placitis Hippocratis et Plotonis} 2.5, 2.7, and by Tideus in \emph{De Speculis}. Though an ancient analogy, Descartes makes distinctively modern use of it. Thus, for example, Descartes not only uses the analogy to motivate his mechanical account of vision but also in support of the claim that there need be nothing in objects that resemble the ideas or sensations that we have of them. Just as the Stoic use of the analogy had its critics, so too the Cartesian use. Thus Merleau-Ponty complains:
\begin{quote}
	The blind, says Descartes, ‘see with their hands’. Cartesian concept of vision is modeled after the sense of touch. At one swoop, then, he removes action at a distance and relieves us of that ubiquity which is the whole problem of vision (as well as its peculiar virtue). \citep[170]{Merleau-Ponty:1964aa}
\end{quote}

% Bibligography
\bibliographystyle{plainnat} 
\bibliography{Philosophy} 

\end{document}
